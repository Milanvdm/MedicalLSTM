\graphicspath{ {Implementation/Images/} }


\chapter{Validation}
\label{cha:implementation}

\section{Introduction}
In this chapter 

DiseaseMapping (generalization)
OSIM
Clusters
TensorFlow
DL4J


\section{Dataset}

To validate the approaches mentioned in chapter \ref{cha:background}, we used a dataset generated by OSIM2. This dataset is used by OMOP to validate their methods to predict the effects of drug treatments. It contains around $10$ million of hypothetical patients based on Thomson Reuters MarketScan Lab Database (MSLR). MSLR contains administrative claims between 2003 ad 2009 from a privately-insured population. \\

The OSIM2 dataset is contains multiple database tables which are dumped as comma-separated values (csv) files. To make it easier to work with this dataset, we joined the multiple files into one file with on each row an event of a patient containing all relevant information. The relevant information which is kept is: birth year, gender, condition type, condition, time difference since previous diagnosis, and season (summer, fall, winter, spring). \\	

Using our approaches on this dataset, we can compare our results to the found clusters in Anders Boeck Jensen et. \cite{Brunak:article}. Although these found clusters are not a golden standard, it is a first validation point for our approaches.


\section{Software}

\subsection{TensorFlow}

TensorFlow is a machine learning software library released at the end of 2015. It is developed by the Google Brain Team. \\




\subsection{DeepLearning4Java}



\section{Experiment Setup}

-generalization
-mapping

-what to test

\section{Results}


\section{Conclusion}


http://omop.org/OSIM2


%%% Local Variables: 
%%% mode: latex
%%% TeX-master: "thesis"
%%% End: 
