\graphicspath{ {Implementation/Images/} }


\chapter{Implementation}
\label{cha:implementation}

\section{Introduction}
In this chapter 

OSIM
Clusters
TensorFlow
DL4J


\section{Dataset}

To validate the approaches mentioned in chapter \ref{cha:approach}, we used a dataset generated by OSIM2. This dataset is used by OMOP to validate their methods to predict the effects of drug treatments. It contains around $10$ million of hypothetical patients based on Thomson Reuters MarketScan Lab Database (MSLR). MSLR contains administrative claims between 2003 ad 2009 from a privately-insured population. \\

The OSIM2 dataset is contains multiple database tables which are dumped as comma-separated values (csv) files. To make it easier to work with this dataset, we joined the multiple files into one file with on each row an event of a patient containing all relevant information. \\

Using our approach on this dataset, we can compare our results to the found clusters in Anders Boeck Jensen et. [DANISH PAPER].


\section{Software}

\subsection{TensorFlow}

TensorFlow is a machine learning software library released at the end of 2015. It is developed by the Google Brain Team. \\




\subsection{DeepLearning4Java}

As mentioned in section \ref{sec:word2vec}, a trained 2-layer neural network can be placed before another neural network and function as a lookup table. In this section, we discuss a possible neural network which allows us to further investigate the effectiveness of our word2vec approach to classify patients. More concrete: we should check if a better accuracy is acquired with the lookup table in front of the neural network or without. 



\section{Conclusion}
The final section of the chapter gives an overview of the important results
of this chapter. This implies that the introductory chapter and the
concluding chapter don't need a conclusion.


http://omop.org/OSIM2


%%% Local Variables: 
%%% mode: latex
%%% TeX-master: "thesis"
%%% End: 
