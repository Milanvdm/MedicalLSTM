\chapter{Conclusion}
\label{cha:conclusion}

This thesis started with explaining the new research area of Electronic Health Record Analytics. We explored the possible impact of this area as it allows to find medical patterns on a large scale. Those patterns range from drug discovery, disease progression for individuals, and reducing medical costs. At the moment several research groups are working on utilizing EHRs to find medical patterns using several methods like querying, statistics, data mining, and artificial intelligence approaches. \\
Our research sought to explore the usage of new machine learning approaches to find correlations between different diagnoses. The correlations found can be used in combination with prediction or classification methods. \\

We introduced a new way on how to apply Word2Vec methods by explaining the link between sentences of words and sequence of medical records. We call this approach a generalized Word2Vec approach and it can be applied on medical data to the correlations between different diagnoses. \\
To make sure the generalized Word2Vec methods can be applied to large-scale medical data, we applied the generalization concept also on DeepWalk. DeepWalk makes it possible to generate a smaller dataset from the original dataset and then apply a Word2Vec approach on this smaller dataset. \\
Besides the exploration on generalizing Word2Vec approaches, we also improved Word2Vec by tackling on of its shortcomings. This shortcoming of Word2Vec is that it is unable to handle unseen instances once it has built his lookup table. We combine a k-nearest neighbors method with Word2Vec and make an estimation of the correlation to other diagnoses for the unseen instance. \\

We explained how we built a model of our proposed methods. We introduced the OSIM2 dataset on which we trained a model using our generalized Word2Vec, knn Word2Vec, and generalized DeepWalk. \\
We explained our categorization approach using fixed intervals and disease code mapping. With this categorization of the data we enabled more general EHR events during training. We also explained the different parameters for each of our approaches and discussed why we chose their corresponding values. \\

After building the model, we executed two different experiments by generating clusters based on our model. The reason behind the clusters is that we could compare those clusters with the clusters found in the Danish paper. In a sense, experiment $2$ looks at how well the relations are for all the diseases in the Danish clusters and not only the diseases which are in the lookup table of the Word2Vec model as done in experiment $1$. We quantify this comparison using a matching percentage.  \\
We checked the influence of a parameter on the matching percentage by inspecting each parameter setting. From all these parameter settings, we deduced general trends. \\
After checking the individual parameters, we took the parameter setting with the highest average matching percentage for each approach and compared the different approaches. \\

The results from the experiment are that both experiments for each approach have a maximum matching percentage of $60$\%. It is still difficult to quantify how well a match is of around $60$\% but we concluded this is good enough to show the potential of our approaches. \\

We conclude that both our knn Word2Vec and DeepWalk have the same performance as the basic generalized Word2Vec. This means that we can handle unseen EHR events and train our model on $50$\% of the dataset size using DeepWalk without losing accuracy. \\
Within the limitations of our validation method, we conclude that our models do match the Danish results well depending on the experiment. Especially since we use several estimations such as the disease code mapping, different datasets, and categorization. \\ 

For more information about another possible experiment which also combines the Word2Vec approaches with prediction or classification methods, we refer to the next chapter. In this chapter we also talk about some possible improvements and future directions to explore.

%%% Local Variables: 
%%% mode: latex
%%% TeX-master: "thesis"
%%% End: 
