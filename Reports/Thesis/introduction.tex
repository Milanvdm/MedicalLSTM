\chapter{Introduction}
\label{cha:introduction}

Grosso modo zou dit ongeveer de opbouw moeten/kunnen zijn:
-          Algemene context (meer en meer data wordt verzameld in electronic health records. Dat is nuttig voor dokters, patienten en de overheid, de analyse is echter nog maar het begin en staat in zijn kinderschoenen (dat begint dan te leiden naar problemen: waarom die analyse nog maar begint is omdat ze moeilijk is à problemen om op te lossen)
-          Dan het algemeen probleem dat je wil oplossen: je wil analyse doen van gegevens uit EHR records. Er is al een beetje werk rond gebeurd (kort iets zegen over Deense paper), maar nog niet veel
-          Dan de “technisch/wetenschappelijke” uitdagingen die je moet oplossen om het probleem aan te pakken: je hebt time series die je moet analyseren, je moet die formaten kunnen vergelijken, etc.
-          Lijst van je contributies,
-          De validatie
-          De conclusie,
-          Een paragraag met de struktuur van de rest van de tekst (in Chapter 2 we will give background information on electronic health record analysis. In …)

This thesis finds itself in the context of the new research area of Electronic Health Record Analytics. In this area a large amount of medical data is available which can be used to find patterns between several diagnoses. At the moment several research groups are working on utilizing EHRs to find medical patterns using several methods like querying, statistics, data mining, and artificial intelligence approaches. Those patterns can be used to improve the accuracy of prediction or classification methods which have as goal to improve personal medical care, drug discovery, disease progression, or reduce medical costs. \\

Based on the analogy between sentences of words and sequences of medical events, we introduce generalized Word2Vec. This generalized Word2Vec makes it possible to be applied on medical data. \\
To make sure the generalized Word2Vec methods can be applied to large-scale medical data, we apply the generalization concept also on DeepWalk. DeepWalk makes it possible to generate a smaller dataset from the original dataset and then apply a Word2Vec approach on this smaller dataset. \\
Besides the exploration on generalizing Word2Vec approaches, we also improve Word2Vec by tackling one of its shortcomings. This shortcoming of Word2Vec is that it is unable to handle unseen instances once it has built his lookup table. We combine a k-nearest neighbors method with Word2Vec and make an estimation of the correlation to other diagnoses for the unseen instance. \\

We compare our results from our Word2Vec methods applied on the OSIM2 dataset to the results of the currently largest study on EHRs. To make it possible to apply our methods on the OSIM2 dataset, we had to categorize our features and make a disease code mapping between two standards namely MedDRA and ICD10. 

%%% Local Variables: 
%%% mode: latex
%%% TeX-master: "thesis"
%%% End: 
